\documentclass{article}
\usepackage[utf8]{inputenc}
\usepackage[brazil]{babel}
\usepackage{tabularx}
\begin{document}


\title{Documentação Trabalho Prático 2}
\author{Gabriel Castelo Branco Rocha Alencar Pinto, Universidade Federal de MInas Gerais}

\date{11 de Dezembro de 2022}
\maketitle

\section{Introdução}
Neste trabalho prático, foi abordado o famoso "Problema do Caixeiro Viajante" (do inglês, Travelling Salesman Problem, ou TSP). O problema, em linhas gerais, consiste no seguinte: Um vendedor, saindo de uma determinada cidade, deseja passar por todas as cidades em um local, passando apenas uma vez em cada cidade. O objetivo do problema é buscar o menor caminho possível que cumpra tais condições. \\ O TSP é, notávelmente, um problema \textit{NP-Difícil}, ou seja, um algoritmo que resolva o problema terá fator de dificuldade no mínimo exponencial. O foco deste trabalho, portanto, será utilizar duas diferentes abordagens para calcular (ou estimar) o caminho mínimo. São elas:

\begin{itemize}
    \item Algoritmo Branch-And-Bound, como solução exata para o TSP, mas com custo exponencial
    \item Algoritmos aproximativos (mais especificamente, os algoritmos de Christofides e Twice-Around-The-Tree). 
\end{itemize}
Além disso, foi parte do trabalho a criação de um gerador de instâncias do problema. O gerador é responsável por construir instâncias de tamanho $2^i$, $\forall{i}\ 4 \le i \le 10$. Uma instância é composta por $i$ pares ordenados $(x, y)$ estritamente diferentes, que representam pontos em um plano cartesiano. 

\section{Detalhes de Implementação}
Para trabalhar com a maior eficiência possível, o trabalho foi implementado em C++11, com o compilador GNU g++. Para representação das estruturas lógicas (grafos e matrizes de adjacência) a biblioteca iGraph (versão 0.10.2) foi utilizada. Além disso, devido à ausência de algoritmos para cálculo do Matching de peso mínimo em grafos na biblioteca iGraph, utilizado no algoritmo de Christofides, foi utilizado um script Python, no padrão Python 3.9+. Dentro deste, foram utilizadas as Bibliotecas Networkx e Numpy. \\
Para Facilitar a interpretação do problema em termos de códigos, foram criadas três classes, para fazer as 3 principais funcionalidades exigidads pela situação:
\begin{itemize}
    \item \textbf{instance-generator}: Classse responsável por gerar as instâncias, garantindo que cada ponto seja único no espaço, além de salvar em um arquivo de texto o resultado.
    \item \textbf{instance-processor}: Classe Responsável por processar um arquivo de texto com uma instância, gerando um vetor com os pontos da mesma.
    \item \textbf{solver}: Classe responsávelpor resolver a instâcia, gerando a matriz de adjacência, calculando as distâncias de Euclides e Manhattan de todos os pontos para todos os pontos, e, por fim, Aplicando os algoritmos para cada uma das 3 instâncias.
\end{itemize}
Estas classes consituem o corpo do programa. Agora, tratando especificamente dos algoritmos utilizados:

\begin{itemize}
    \item \textbf{Branch And Bound}: A implementação utilizada para o algoritmo Branch-And-Bound foi uma transposição direta do pseudocódigo apresentado em sala de aula para código C++. Para esta implementação, não foram utilizadas as estruturas para tratar de grafos baseadas na biblioteca \textit{iGraph}, mas sim estruturas padrão implementadas na \textit{Standart Template Library}, a STL de C++.
    \begin{itemize}
        \item O custo do Algoritmo Branch And Bound é, no pior caso, exponencial em essência, pois visitaria todas as soluções possíveis até encontrar a máxima. Todavia, é uma garantia de segurança para a resposta, em casos onde não se pode tolerar a margem de erro dos algoritmos aproximativos.
        \item As principais estruturas de dados utilizadas foram: uma priority queue, para armazenar os pontos a serem visitados, e a estrutura \textbf{Node}, que basicamente é um container que possui 4 elementos: \textit{Bound, cost, level e s}. O valor S é um vetor contendo id dos demais nós com os quais o elemento faz fronteira.
    \end{itemize}
    \item \textbf{Twice Around The Tree}: A implementação do algoritmo Twice around the Tree foi feita seguindo as orientações das notas de aula, todavia, para representar as estruturas de dados e para computar algoritmos secundários, como uma \textit{Árvore Geradora Mínima}, ou MST, foi utilizada a biblioteca \textit{iGraph}. Contendo uma grande diversidade de algoritmos, a biblioteca permite representar rapidamente um grafo, dado sua matriz de adjacência, calcular árvore geradora mínima e as demais necessidades do algoritmo.
    \begin{itemize}
        \item O algoritmo Twice Around the Tree é 2-aproximativo para o problema do caixeiro viajante, ou seja, enquanto que ele não provêm necessariamente a melhor solução, chegando a propor uma solução até duas vezes pior para o problema a depender da instâcia, seu custo de execução é polinomial.
    \end{itemize}
    \item \textbf{Algoritmo de Christofides}: O algoritmo de Christofides, ao contrário dos Demais, não foi implementado em C++, devido às limitações da biblioteca \textit{iGraph}. foi definido, portanto, um script auxiliar em Python para a implementação do mesmo. As estruturas utilizadas para execução do programa são aquelas providas pela biblioteca \textit{Networkx}. 
    \begin{itemize}
        \item O algoritmo de Christofides é um algoritmo 1.5 aproximativo, ou seja, um algoritmo que, no pior caso, dará uma solução 1.5x pior do que aquela considerada a ótima. Nota-se, portanto, que é um algoritmo estritamente melhor do que o Algoritmo Twice Around the Tree, ou seja, apresenta resultados sempre menores. Tal assertiva é confirmada mais adiante na sessão 3 deste estudo.
    \end{itemize}
\end{itemize}

\section{Experimentos}
Aqui, serão demonstrados os resultados dos testes com as instâncias geradas conforme explicado na sessão 1. Nota-se que há diferença considerável no tempo de execução entre o algoritmo de Christofides e o Twice Around The Tree, todavia tal atraso se deve à implementação deste ser em Python\footnote{Por ser uma Linguagem Interpretada, o custo de tempo de tokenização é relevante no que tange a algoritmos custosos.}, enquanto que os outros foram escritos em C++, compilados com otimização máxima permitida pelo compilador (-O3)\footnote{As configurações da máquina utilizadas para o teste são: Processador Ryzen 5 5600G, Memória: 16GB DDR4 3000MHz, GPU AMD RX 6700XT.}. \\
O algoritmo Branch And Bound, devido ao custo de execução, acabou por não produzir resultados em tempo hábil para esta demonstração.\footnote{Respeitando o limite de tempo de 30 minutos.}

\begin{tabular}{ |p{5cm}||p{2cm}|p{3cm}|p{3cm}|}
 \hline
 \multicolumn{4}{|c|}{Resultados do Experimento} \\
 \hline
 Algortimo & Instância & Tempo de Execução & Caminho Mínimo\\
 \hline
 Christofides & $2^4$ & 0.244575 s & 261.689980\\
 Christofides & $2^5$ & 0.136870 s & 357.067330 \\
 Christofides & $2^6$ & 0.140664 s & 549.668480\\
 Christofides & $2^7$ & 0.196704 s & 769.458530\\
 Christofides & $2^8$ & 0.333948 s & 1032.053680\\
 Christofides & $2^9$ & 1.010247 s & 1498.529060\\
 Christofides & $2^10$& 4.053145 s & 2135.295680\\
 Twice Around the Tree & $2^4$ & 0.000044 s & 932.646238\\
 Twice Around the Tree & $2^5$ & 0.000091 s & 1579.482045\\
 Twice Around the Tree & $2^6$ & 0.000308 s & 4162.018635\\
 Twice Around the Tree & $2^7$ & 0.001305 s & 7806.920033\\
 Twice Around the Tree & $2^8$ & 0.006007 s & 17041.619342\\
 Twice Around the Tree & $2^9$ & 0.028102 s & 30895.474565\\
 Twice Around the Tree & $2^10$& 0.154774 s & 66224.750684\\
 Branch And Bound & $2^4$ & N/A  & N/A\\
 Branch And Bound & $2^5$ & N/A & N/A\\
 Branch And Bound & $2^6$ & N/A & N/A\\
 Branch And Bound & $2^7$ & N/A & N/A\\
 Branch And Bound & $2^8$ &N/A & N/A\\
 Branch And Bound & $2^9$ & N/A & N/A\\
 Branch And Bound & $2^10$& N/A & N/A\\
 \hline
\end{tabular}

\section{Conclusões Finais}

Após a execução dos algoritmos e subsequente análise de resultados, diversos fatos ficam evidentes. A priori, é possível perceber a diferença de eficiência de cada algoritmo aproximativo com relação aos valores de caminho mínimo encontrados. É evidente que o algoritmo de Christofides é bem mais eficiente no que tange à precisão do resultado, encontrando inexoravelmente soluções menores do que o Twice Around the Tree. Todavia, mesmo considerando as diferenças de plataformas que separam Python e C++, fica evidente que o custo do algoritmo de Christofides, que envolve não somente o cálculo de uma MST, mas também um Matching de Peso mínimo, pesa no quesito temporal. \\ 
Destas evidências, fica demonstrado como o custo teórico se aplica ao em situações práticas, ou seja, em termos de tempo de execução e acuidade de resultados.

\section{Instruções de Compilação}
O programa foi desenhado para funcionar especificamente em máquinas Linux, executando comandos que não funcionariam em outros sistemas operacionais. Os testes foram realizados em um Ubuntu 20.04. Faz-se necessária, também, a existência da biblioteca iGraph instalada, compilada estaticamente. Por fim, o interpretador Python3 e a Biblioteca Networkx devem ambos estarem instalados. \\
Para Compilar o programa, basta acessar o diretório da pasta já descompactada e esecutar o comando \textit{make}.\\
As instruções para execução do programa podem ser obtidas através de \textit{./main -h}.

\end{document}
